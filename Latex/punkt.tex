\section{Punkte}
Ein Punkt ist ein dreidimensionaler Vektor.

\begin{equation}
	\vec{p} = \myvec{p_{x} \\ p_{y} \\ p_{z}}
\end{equation}

\section{Linien}
Linien $(\vec{L})$ haben jeweils einen Anfangs- und einen Endpunkt, $\vec{p}_{0}$ und $\vec {p}_{1}$. Die Linie wird durch die Menge aller Punkte $\vec{q}$ beschrieben, die folgender Gleichung genügen:

\begin{equation*}
	\vec{q} = \vec{p}_{0} + \lambda (\vec{p}_{1} - \vec{p}_{0})
\end{equation*}
Durch Einsetzen der verkürzten Schreibweise $\vec{p}_{i} - \vec{p}_{k} = \vec{p}_{ik}$ ergibt sich:
\begin{equation}
	\boxed{ \vec{q} = \vec{p}_{0} + \lambda \cdot \vec{p}_{10} } \quad \lambda \in [0; 1]
\end{equation}